%-------------------------
% Resume in Latex
% Author : Jake Gutierrez
% Based off of: https://github.com/sb2nov/resume
% License : MIT
%------------------------

\documentclass[letterpaper,11pt]{article}

\usepackage{latexsym}
\usepackage[empty]{fullpage}
\usepackage{titlesec}
\usepackage{marvosym}
\usepackage[usenames,dvipsnames]{color}
\usepackage{verbatim}
\usepackage{enumitem}
\usepackage{xcolor}
\usepackage[colorlinks=true, linkcolor=darkblue, urlcolor=darkblue, citecolor=darkblue]{hyperref}
\definecolor{darkblue}{RGB}{0, 0, 139} % Darker than standard blue
\usepackage{fancyhdr}
\usepackage[french]{babel}
\usepackage{tabularx}
\usepackage{fontspec}
\defaultfontfeatures{Ligatures=TeX}


%----------FONT OPTIONS----------
% sans-serif
% \usepackage[sfdefault]{FiraSans}
% \usepackage[sfdefault]{roboto}
% \usepackage[sfdefault]{noto-sans}
% \usepackage[default]{sourcesanspro}

% serif
% \usepackage{CormorantGaramond}
% \usepackage{charter}


\pagestyle{fancy}
\fancyhf{} % clear all header and footer fields
\fancyfoot{}
\renewcommand{\headrulewidth}{0pt}
\renewcommand{\footrulewidth}{0pt}

% Adjust margins
\addtolength{\oddsidemargin}{-0.5in}
\addtolength{\evensidemargin}{-0.5in}
\addtolength{\textwidth}{1in}
\addtolength{\topmargin}{-.5in}
\addtolength{\textheight}{1.0in}

\urlstyle{same}

\raggedbottom
\raggedright
\setlength{\tabcolsep}{0in}

% Sections formatting
\titleformat{\section}{
  \vspace{-4pt}\scshape\raggedright\large
}{}{0em}{}[\color{black}\titlerule \vspace{-5pt}]

%-------------------------
% Custom commands
\newcommand{\resumeItem}[1]{
  \item\small{
    {#1 \vspace{-2pt}}
  }
}

\newcommand{\resumeSubheading}[4]{
  \vspace{-2pt}\item
    \begin{tabular*}{0.97\textwidth}[t]{l@{\extracolsep{\fill}}r}
      \textbf{#1} & #2 \\
      \textit{\small#3} & \textit{\small #4} \\
    \end{tabular*}\vspace{-7pt}
}

\newcommand{\resumeSubheadingDouble}[6]{
  \vspace{-2pt}\item
    \begin{tabular*}{0.97\textwidth}[t]{l@{\extracolsep{\fill}}r}
      \textbf{#1} & #2 \\
      \textit{\small#3} & \textit{\small #4} \\
      \textit{\small#5} & \textit{\small #6} \\
    \end{tabular*}\vspace{-7pt}
}

\newcommand{\resumeSubSubheading}[2]{
    \item
    \begin{tabular*}{0.97\textwidth}{l@{\extracolsep{\fill}}r}
      \textit{\small#1} & \textit{\small #2} \\
    \end{tabular*}\vspace{-7pt}
}

\newcommand{\resumeProjectHeading}[3]{
\vspace{-2pt}\item
    \begin{tabular*}{0.97\textwidth}{l@{\extracolsep{\fill}}r}
      \textbf{#1} & #2 \\
      \textit{\small#3} & \space \\
    \end{tabular*}\vspace{-7pt}
}

\newcommand{\resumeSubItem}[1]{\resumeItem{#1}\vspace{-4pt}}

\renewcommand\labelitemii{$\vcenter{\hbox{\tiny$\bullet$}}$}

\newcommand{\resumeSubHeadingListStart}{\begin{itemize}[leftmargin=0.15in, label={}]}
\newcommand{\resumeSubHeadingListEnd}{\end{itemize}}
\newcommand{\resumeProjectListStart}{\begin{itemize}[leftmargin=0.15in, label={}]}
\newcommand{\resumeProjectListEnd}{\end{itemize}}
\newcommand{\resumeItemListStart}{\begin{itemize}}
\newcommand{\resumeItemListEnd}{\end{itemize}\vspace{-5pt}}

%-------------------------------------------
%%%%%%  RESUME STARTS HERE  %%%%%%%%%%%%%%%%%%%%%%%%%%%%


\begin{document}

\begin{center}
    \textbf{\Huge \scshape Dion Osmani}
\end{center}


%-----------EDUCATION-----------
\section{Éducation}
  \resumeSubHeadingListStart
    \resumeSubheading
      {HES-SO, Haute École d'Ingénierie}{Sion, VS}
      {BSc Informatique et Systèmes de communication, ingénierie des données}{Août 2022 -- Sep. 2025}
    \resumeSubheadingDouble
      {École Professionnelle Technique et des Métiers}{Sion, VS}
      {CFC Informaticien, systèmes et réseaux}{Août. 2018 -- Juin 2022}
      {Maturité technique}{"}
  \resumeSubHeadingListEnd


%-----------Experience-----------
\section{Expérience}
  \resumeSubHeadingListStart

    \resumeSubheading
      {HES-SO, Infrastructure Competency Center}{Août 2021 -- Jui. 2022}
      {Stagiaire}{Sion/Sierre, VS}
      \resumeItemListStart
      \resumeItem{Designé et développé une dénomination et catégorisation automatique des machines virtuelles d'un environnement vSphere.}
      \resumeItem{Développé une application web progressive pour faciliter la création de machines virtuelles par le corps métier et ainsi réduire la charge de travail des administrateurs système.}
      \resumeItemListEnd

  \resumeSubHeadingListEnd



%-----------Extracurriculars-----------
\section{Activités Annexes}
  \resumeSubHeadingListStart

  \resumeSubheading
    {Promotion des métiers}{Août 2022 --}
    {Animateurs}{VS}
    \resumeItemListStart
        \resumeItem{Présenté les opportunités de formation permettant d’accéder au niveau universitaire, en expliquant les parcours possibles aux jeunes.}
        \resumeItem{Sensibilisé les élèves à l’intelligence artificielle, en
        exposant l’état actuel du domaine et ses applications concrètes.}
        \resumeItem{Assuré la promotion de la filière auprès d’un public jeune, en animant des stands et en échangeant sur les métiers du secteur.}
    \resumeItemListEnd

    \resumeSubheading
      {Communication HES-SO Valais/Wallis}{Août 2024 -- Sep. 2025}
      {Ambassadeur}{VS}
      \resumeItemListStart
      \resumeItem{Créé du contenu visuel et vidéo pour Instagram et TikTok, en lien avec la vie étudiante et les activités sur le campus universitaire.}
      \resumeItemListEnd

  \resumeSubHeadingListEnd

% -----------Skill and Interests-----------
% \section{Qualités \& Intérêts}
%  \begin{itemize}[leftmargin=0.15in, label={}]
%     \small{\item{
%     \vspace{1mm}
%      \textbf{Compétences}{: } \\
%      \vspace{1mm}
%      \textbf{Intérêts}{: Photography, Creative Writing, Archery, New York Knicks} \\
%      \vspace{1mm}

%     }}
%  \end{itemize}

% Projects
\section{Projets}
\resumeProjectListStart
\resumeProjectHeading
    {Adaptive Mesh Refinement}{\href{https://github.com/dij0s/AMR}{github.com/dij0s/AMR}}
    {Projet académique}
    \resumeItemListStart
        \resumeItem{Développé un algorithme d'adaptation dynamique de maillage (AMR) en Python, optimisant les ressources de calcul avec une réduction de 89\% du temps d'exécution.}
        \resumeItem{Implémenté une structure de données Quadtree/Octree efficace pour la simulation numérique, avec une réduction de 80\% de l'utilisation mémoire.}
        \resumeItem{Résolu des équations de diffusion thermique avec un schéma aux différences finies du second ordre, validé par comparaison avec une solution de référence.}
        \resumeItem{Appliqué des pratiques de développement professionnel incluant des tests unitaires étendus, l'intégration continue (GitHub Actions) et une documentation technique détaillée.}
    \resumeItemListEnd
    \vspace{1mm}

\resumeProjectHeading
    {Autopilot Driver Genetic Algorithm}{\href{https://github.com/dij0s/AdGA}{github.com/dij0s/AdGA}}
    {Projet académique}
    \resumeItemListStart
    \resumeItem{Conçu et implémenté un framework distribué d'optimisation de trajectoires pour pilote automatique utilisant des algorithmes génétiques.}
           \resumeItem{Développé un système de parallélisation massive sur supercalculateur via SLURM, permettant l'optimisation simultanée de multiples trajectoires.}
           \resumeItem{Développé une heuristique d'évaluation des trajectoires basée sur les principes de conduite automobile optimale.}
           \resumeItem{Mis en place une architecture évolutive permettant le passage à l'échelle des calculs d'optimisation sur des clusters de calcul à l'aide de Kubernetes.}
           \resumeItem{Implémenté des algorithmes génétiques pour l'optimisation multi-objectifs des trajectoires de conduite autonome.}
    \resumeItemListEnd
\resumeProjectListEnd

\vspace{5mm}

\begin{center}
\small +41 78 742 1190 $|$
\href{mailto:dion.08osmani@gmail.com}{dion.08osmani@gmail.com} $|$
\href{https://www.linkedin.com/in/dion-osmani-60478720b/}{linkedin.com} $|$
\href{https://github.com/dij0s/}{github.com}
\end{center}


\end{document}
